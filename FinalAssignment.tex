\documentclass{article}

\usepakage{xcolor}

\title{Final Assignment}
\author{Ehsan Abdollahi Nasab}
\date{\today}

\begin{document}

\maketitle

\section{Git & GitHub}
\subsection{Repository Initialization and Commits}
\textbf{Write about how you set up the repository for this assignment. Explain every step in detail.}
\textcolor{green}{For this I used GitHub website. I chose "Create New..." then "New Repository". After that I set the Repository; named it "FinalAssignment", chose it to be "Public" and add a "README file".}

\subsection{GitHub Actions for LaTeX Compilation}
\textbf{Provide a walkthrough of setting up GitHub Actions to automatically compile your LaTeX document and any challenges you encountered.}
\textcolor{green}{}

\section{Exploration Tasks}
\subsection{Vim Advanced Features}
\textbf{Explore and document 3 advanced features of Vim that were not covered in class.}
\textcolor{green}{}
\subsection{Memory Profiling}
\textbf{This semester, you got to know about dynamic memory allocation in C in your Programming Fundamentals class.}
\textcolor{green}{}
\subsubsection{Memory Leak}
\textbf{In short, explain what memory leaks are and how they might happen in your program.}
\textcolor{green}{}
\subsubsection{Memory Profilers}
\textbf{Read about a tool called Valgrind and write about their purpose and how it helps when memory leaks happen.}
\textcolor{green}{}

\section{GNU/Linux Bash Scripting}
\subsection{fzf}
\textbf{What is fuzzy searching? Give a short description.}
\textcolor{green}{}
\textbf{Install fzf on your machine and give a description of what the following command does:
ls | fzf}
\textcolor{green}{}
\subsection{Using fzf to find your favorite PDF}
\textbf{You might have came across moments when you want to open up a certain PDF when studying for your final exams but finding the directory of that PDF is a very gruesome and tiring process. In this section, we will be using fzf to find our PDF in seconds! We will be going step by step on how to find your file and use fzf to select it.
\begin{enumerate}
    \item We first need to list the directory of all the files with the extension .PDF. Write a command to list the directory of all the files with the extension .PDF
    \item Now we have to select the PDF we want using fzf. Write a command to use fzf to select a PDF from the data we gathered above.\end{enumerate}}
\textcolor{green}{}
\subsection{Opening the file using Zathura}
\textbf{Now that have selected which PDF we want to open, we can use a very minimalistic program called Zathura to open it. Write a command that uses the commands above to open the file using Zathura.}

\section{Git and FOSS}
\subsection{README.md}
\textbf{Make sure to include a basic README.md file in your GitHub repository that describes the aim of this repository and its purpose. 
Make sure to use headings and lists in your README.md file.}
\textcolor{green}{}
\subsection{Issues}
\textbf{Create a sample issue in the repository below and attach its screenshot in your LaTeX document: https://github.com/MiliAxe/CW-Final}
\textcolor{green}{}
\subsection{FOSS contribution}
\textbf{Do you see yourself contributing to FOSS projects in the future? If yes, what kind of projects are you interested in contributing to? If no, why not?}
\textcolor{green}{}

\end{document}